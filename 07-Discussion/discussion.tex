\chapter{General discussion}
\label{chap:general-discussion}

The aim of this chapter is to bring together the different aspects of the research carried out (Chapters \ref{chap:eodal}-\ref{chap:drc}) within the framework of a landscape-scale prototype and to discuss its applications, limitations and possible further research questions. However, the scientific discussion of the individual research chapters is not repeated here. Please refer to the relevant discussion subsections in the previous chapters.

\section{A prototype for landscape-scale phenotyping}
Figure \ref{fig:oa-disc-prototype} shows a sketch of how the individual research components can be transferred to a landscape level prototype to quantify winter wheat growth and development.

The main data sources for the prototype are environmental covariates, mainly air temperature, and high resolution optical satellite imagery from the \gls{S2} mission. The necessary calibration of the prototype is mainly based on field phenotyping data, which encode the relationships between development and growth (see Chapter \ref{chap:insights}) and between plant growth and environmental conditions (Chapter \ref{chap:drc}). These calibration data thus represent the physiological and phenological knowledge of G $\times$ E interactions in crops in general and wheat in particular.

\subsection{Components}
Using the calibration and the two data sources, the functionality of the prototype can now be demonstrated in three components.

\paragraph{Components1 -- Timing and duration of key phenological stages}
The first step is to determine the timing of key phenological development stages as described in Chapters \ref{chap:phemology} and \ref{chap:insights}. This is important to determine the onset and duration of the \gls{SE} period, which was the main focus of this thesis due to its importance in grain yield formation, as explained in Chapter \ref{chap:introduction}. The phenology model uses day length and weather data and has a coarse spatial resolution (km scale) due to the relatively coarse resolution of most meteorological data products at the landscape scale. The timing extracted from the phenology model will therefore limit the period over which satellite data should be considered.

\paragraph{Component 2 -- Trait retrieval from satellite imagery}
Once the relevant time period has been extracted from the phenology model, satellite data are searched and converted to \gls{GLAI} using physiological and phenological priors from field phenotyping as described in Chapter \ref{chap:insights} using RTM inversion and the software \gls{EOdal} (Chapter \ref{chap:eodal}). Thus, at $n$ times, where $n$ is the number of cloud-free \gls{S2} scenes, \gls{GLAI} estimates are available at 10 $\times$ 10 m spatial resolution. This allows spatial detail to be resolved, e.g. on within-field heterogeneity, which, as noted above, is not available from the temperature data. The \gls{GLAI} estimates based on the \gls{S2} data are thus snapshots of the apparent growth conditions.

\paragraph{Component 3 -- Reconstruction of growth}
Using the \gls{S2} \gls{GLAI} observations and air temperature data for a second time, the growth dynamics in the \gls{SE} period can be modelled in hourly or daily resolution in a final step, as described in Chapter \ref{chap:drc} using \gls{DRC}s. This combines the high temporal resolution of the temperature data is combined with the spatial detail of the \gls{S2} \gls{GLAI} observations. Not only is the physiological knowledge from the field phenotyping encoded here, but the uncertainty propagation performed in Chapter \ref{chap:uncertainty} is also used to model growth and development during the \gls{SE} period.

\begin{figure}[H]
    \centering
    \includegraphics[width=\textwidth]{07-Discussion/img/prototype.jpg}
    \caption{The proposed prototype for landscape scale phenotyping of winter wheat growth and development as a key outcome of this thesis.}
    \label{fig:oa-disc-prototype}
\end{figure}

The prototype composed of these three components thus allows the quantification of winter wheat growth and development at the landscape scale, for example in the Swiss Plateau, with a spatial resolution of up to 10 m and a temporal resolution of up to one hour. Furthermore, by combining satellite, meteorological and in-situ data, the prototype can be considered as a true \gls{EO} system. At the same time, the quantification of growth through \gls{GLAI} values and development through phenological stages means that the prototype fulfils the definition of a phenotyping system.

\subsection{Applications}
The prototype can effectively address the two pathways of agricultural transformation mentioned in Section \ref{sec:intro-motivation} to overcome the current challenges of negative environmental impacts of agricultural production, while helping to meet the increased demand for food and biomass.

\paragraph{Decision support and policy advice}
The phenotyping prototype presented is in line with multilateral initiatives such as \gls{GEOCLAM} or the European Union's Joint Research Centre MARS Bulletin with its European Crop Monitor \citep{van_der_velde_use_2019}, which aim to provide traceable, actionable insights to stakeholders in agriculture \citep{whitcraft_no_2019}. As the prototype is based on open satellite and, at least in most cases, readily available temperature data and rather simple but physiologically meaningful models, it appears well suited for further operationalisation and near real-time delivery of agronomically relevant information. Near-real-time information on the development and growth of stable crops such as wheat is arguably crucial for informed decision making, e.g. to reduce excessive fertiliser run-off \citep{argento_linking_2022} or for policy advice on potential crop failures and food shortages \citep{becker-reshef_strengthening_2020}. In this way, the scientific efforts of this thesis could contribute to increasing the resilience of the agricultural sector as a whole.

\paragraph{Research and breeding}
The phenotyping prototype also has potential applications in crop science, breeding and variety testing: in the future, in addition to small-scale field phenotyping trials, experiments could be conducted at the landscape scale on ``real'' farms to better investigate effects such as soil, topography or microclimate. This is in line with the objectives of the European project EMPHASIS\footnote{\url{https://emphasis.plant-phenotyping.eu/}}, which aims to establish a pan-European, cross-scale phenotyping infrastructure for more sustainable and resilient food production \citep{pieruschka_plant_2019}. Landscape-scale phenotyping could not only deepen the scientific understanding of plant-environment interactions, but also contribute to accelerated variety testing and provide breeders with an additional tool for selecting and evaluating breeding lines.

\section{Answers to research questions}
With the prototype (Figure \ref{fig:oa-disc-prototype}) the three research questions outlined in section \ref{sec:intro-obj-rj} can be addressed.

\subsection{How can field phenotyping and spaceborne remote sensing data be combined to allow up-scaling of physiological knowledge from field phenotyping to the landscape-scale?}
This thesis has identified two ways of combining field phenotyping and spaceborne remote sensing data. The first way is to use field phenotyping data as prior knowledge to constrain \gls{RTM} simulations as shown in chapter \ref{chap:insights}, and to fill temporal gaps and remove outliers using \gls{DRC}s to reconstruct hourly or daily \gls{GLAI} trajectories from single \gls{S2} observations (Chapter \ref{chap:drc}). While the first way directly addresses the workflow of remote sensing retrieval and time series reconstruction, the second way is about using field phenotyping data to parameterise phenological models (Chapter \ref{chap:phemology}), which in turn are used to select relevant satellite imagery and quantify the timing of key developmental stages such as the end of heading. It is thus a more indirect way of incorporating field phenotyping data into an \gls{EO} approach. As a result, both pathways allow knowledge and concepts to be scaled up from field phenotyping to the landscape scale.

\subsection{Can a landscape-scale phenotyping approach provide accurate, physiologically based and traceable insights into winter wheat growth and development?}
The quantification of uncertainties (see Chapter \ref{chap:uncertainty}) fulfils the requirement for traceability (Section \ref{sec:intro-obj-rj}). The integration of prior knowledge from field phenotyping into the \gls{GLAI} retrieval process in step 2, as well as into the parameterisation of \gls{DRC}s in step 3, fulfils the requirement for physiological plausibility (see also the individual scientific discussions in Sections \ref{sec:insights_discussion} and \ref{sec:drc_discussion}). The accuracy of the methodology has been demonstrated using multi-year, independent in-situ data for phenological development (RMSE for heading date: 2 days, Chapter \ref{chap:phemology}) and GLAI (smallest relative error: 13\%, Chapter \ref{chap:drc}). This research question can therefore be answered in the affirmative.

\subsection{What are the potentials but also the limitations and challenges of such a landscape phenotyping approach?}

The prototype allows the study of G $\times$ E interactions that could not be fully addressed by small-scale field phenotyping experiments. These include effects of changes in soil properties or topography that are spatially continuous and affect plant growth and development through soil water availability, exposure to wind and sunlight, or nutrient availability. In addition, the \gls{GLAI} estimates can be converted to biomass \citep{aase_relationship_1978} and -- in perspective -- grain yield, which are arguably important agronomic traits for decision making and policy advice. Accurate modelling of these traits will therefore not only advance the science behind \gls{EO}-based applications for agriculture, but also help to meet the needs of a growing world population.

The main obstacle to this potential is the lack of in situ data: Management data, in particular the sowing date or the variety used, is essential agronomic information that is not yet available on a large scale. In many countries, including Switzerland, the management data that the state is obliged to collect and publish include the main crop, but no further management data. In addition, research data from field phenotyping and variety testing, which are key to calibrating models, are often not publicly available. When they are, the data are often not standardised and poorly documented, so it is not clear how the data were collected and what the uncertainties are.

Another limitation of the prototype is its spatial resolution of currently 10 m. This can cause spectral mixing problems close to field boundaries and make it difficult to resolve small-scale details such as flower strips, hedges or individual trees within a parcel. For small plots, it may not be possible to find a pure pixel, i.e. a pixel that is not affected by spectral mixing. \cite{meier_assessments_2020} found that, at a spatial resolution of 10 m, up to 6.4\% of all parcels (0.49\% of the total agricultural land area) were lost due to the lack of pure pixels, and for up to 50\% of the fields (10.53\% of the total agricultural land area) no site-specific agricultural applications were possible due to the small number of pure pixels ($\le$ 50 pixels). These figures were reported for Bavaria, which has a comparable average field size (1.6 ha) to Switzerland (1.4 ha) and a similar agricultural land use pattern. A solution to this problem could be the use of higher resolution data, such as from the Planet Labs$^{\circledR}$ constellation, with a pixel size of less than 5 m. In Samuel Wildhaber's master's thesis, it was shown that such high resolution data have potential for agricultural applications and could outperform \gls{S2} data when it comes to representing small-scale detail \citep{wildhaber_assessing_2023}. At the same time, unlike \gls{S2}, the data is proprietary and therefore more expensive to acquire. Deep learning based super-resolution may therefore be a more viable option to improve the spatial detail available from \gls{S2}, but research carried out by Julian Neff as part of his Master's thesis suggests that more effort may be required to achieve satisfactory results.

\section{Open questions}
\subsection{Spatial or temporal detail?}
In Switzerland, the \gls{S2} constellation provides a temporal resolution of 3 to 5 days, depending on orbit coverage \citep{pazur_national_2022}. However, prolonged cloud cover can significantly reduce the number of images available, leaving critical developmental stages uncovered. In addition, undetected clouds and shadows add to the image noise, as the delineation of clouds and their shadows is highly uncertain due to spectral mixing effects, as discussed in Chapter \ref{chap:uncertainty}. The assimilation of \gls{DRC}-based hourly or daily growth rates and \gls{S2} observations has been proposed as a solution to this problem (Chapter \ref{chap:drc}). However, the cloudy spring of 2023 showed that even such sophisticated data assimilation schemes are limited by the number of satellite observations.

Data fusion approaches have therefore been proposed in the scientific literature. These include the fusion of different sensor types, such as different optical platforms or optical and \gls{SAR} data \citep[for example]{pipia_fusing_2019, lobert_mowing_2021}. Here, two strategies have been identified: The first is to increase the number of images by including coarse resolution observations from Landsat (30 m spatial resolution), \gls{MODIS} (250 m) or \gls{S3} (300 m) as suggested, for instance, by \cite{zhou_reconstruction_2020}. These data are also freely available, but lack significant spatial detail, as can be seen in Figure \ref{fig:discussion-l9-vs-s2}. Here, a Landsat-9 image at 30 m resolution is contrasted with an S2 image at 10 m resolution taken on the same day in June 2022 over Witzwil in western Switzerland, where the field size is almost 10 times larger than the Swiss average (13 ha, \cite{perich_pixel-based_2023}). In figure \ref{fig:discussion-l9-vs-s2}, it is clearly visible that the fields appear much more blurred at a resolution of 30 m. In addition, \cite{meier_assessments_2020} estimated that at 30 m, over 40\% of the fields no longer have a pure pixel. The second strategy is therefore to use higher resolution data than \gls{S2}, such as the Planet Labs$^{\circledR}$ data \citep[for example]{sadeh_fusion_2021}. However, this comes at the cost of increased financial burden and the need to use proprietary imagery. The fusion of \gls{SAR} data, which is attractive because of the cloud penetrating ability of the microwave radiation emitted, poses the challenge that the radiative properties in this spectral region differ significantly from optical data. Nevertheless, \cite{bai_could_2020} and \cite{villarroya-carpio_sentinel-1_2022} showed that \gls{SAR} coherence, i.e. the correlation coefficient of interferometric phase changes between single \gls{SAR} acquisitions, is correlated with \gls{NDVI} and is useful for agricultural applications.

\begin{figure}[H]
    \centering
    \includegraphics[width=\textwidth]{07-Discussion/img/comparison_l9-s2_witzwil22.png}
    \caption{Comparison of true-colour Landsat-9 30 m ground-sampling distance (GSD) and Sentinel-2 10 m GSD imagery acquired on 2022-06-13 of agricultural areas around Witzwil, Switzerland.}
    \label{fig:discussion-l9-vs-s2}
\end{figure}

It is clear that the use of Landsat or Sentinel-3 data alone is not a realistic option for Switzerland, given the small size of the field plots. The open question, however, is what kind of data source should be used to fill the gaps. Is it really necessary to use expensive, proprietary data with high spatial and temporal resolution, which also increase storage requirements and computing time, or do coarse spatial resolution data such as \gls{S3} not nevertheless contain a certain amount of information about the crops that could be used for assimilation, for example? This ultimately comes down to the research question: What is more important -- spatial detail, temporal resolution, or both? A systematic comparison of the different platforms and sensors and their applicability for landscape phenotyping is certainly needed to provide answers.

\subsection{What are the limiting factors?}
In the Chapters \ref{chap:insights} and \ref{chap:drc} air temperature is considered as the main driver of plant growth \citep{porter_temperatures_1999}. By growth we mean the increase in \gls{GLAI} which -- during the \gls{SE} period -- correlates with the accumulation of dry matter, i.e. dry biomass (see Figure \ref{fig:ww-growth-development} and \cite{aase_relationship_1978}). \cite{monteith_climate_1977} points out that changes in leaf area are mainly controlled by temperature and soil water availability. Situations where wheat growth is water limited may become more relevant in the future due to climate change, even in Switzerland \citep{holzkamper_spatial_2015}. Currently, the prototype does not consider soil water availability, as this covariate is more difficult and costly to measure and is therefore often missing at larger spatial scales. Remotely sensed topsoil moisture \citep{lobell_moisture_2002, sadeghi_optical_2017} and hydrological modelling of surface water and energy balance \citep{penman_natural_1948, priestley_assessment_1972, shuttleworth_simplified_1978} could provide this information at larger spatial scales. However, the question remains whether leaf area dynamics are more controlled by temperature or water availability, as high temperatures tend to decrease soil moisture. This decrease is due to positive, non-linear feedback loops between air temperature, water vapour pressure deficit, evapotranspiration, soil moisture and temperature, and latent and sensible heat fluxes \citep{webber_diverging_2018, garcia-garcia_soil_2023}. Thus, further research on the ecophysiological processes governing the plant-soil-atmosphere continuum to identify the limiting factors for crop growth in terms of \gls{GLAI} dynamics at the landscape scale seems promising to uncover the underlying mechanisms. Another question that, to the best of our knowledge, remains largely unanswered is the effect of water availability on phenology.

An additional environmental covariate not considered in this paper is global radiation, or more precisely the fraction of absorbed \gls{PAR}, FAPAR. \cite{monteith_climate_1977} suggests modelling dry matter accumulation in terms of absorbed radiation and the efficiency with which the input solar radiation is converted into carbohydrates. The amount of absorbed radiation in turn depends on leaf area dynamics, the drivers of which have been outlined above. In an attempt to provide a unified formalism for describing crop growth in terms of leaf area \textsl{and} biomass throughout the crop development cycle \cite{goudriaan_mathematical_1990}, a prototype of what will become light use efficiency models \citep{gitelson_productivity_2015} was developed. Apart from these considerations, too little radiation during meiosis, i.e. the production of gametes with the correct number of chromosomes, could lead to sterile flowers and thus reduce grain yield in winter wheat, which caused significant yield losses in France in 2016 \citep{le_gouis_how_2020}. % phenology AND radiation??


Ultimately, the impact of management such as fertilizing and plant protection measures should not be forgotten. This actually addressed the central question whether M or E are more central on the landscape scale, i.e. if plant growth and development is mainly driven by environmental covariates or farm management decisions. % is there anything?

\subsection{How to represent time?}

% farming practice and breeding use BBCH (or similar) and days after sowing to reference observations
% RS data gives a datetime
% datetime and days after sowing not meaningful due to large year-to-year differences and expected shifts (see Chapter 3)
% challenge is to bring together different time scales and then the question which one to use. Would be nice to have a time scale that has an explicit representation of growth and development, i.e., that not only allows a linearization of growth but that can also be directly linked to import phenological stages such as onset of SE, flowering, and physiological maturity

\section{Outlook}
% extend to senescence phase -> grain filling phase, detection of physiological maturity when GLAI tends to zero
% different crops -> other cereals (barley, maize, rice) but also legumes, etc.
% look into light use efficiency models to simulate biomass and grain filling -> grain yield as the most important agronomic variable
% ultimately: couple with climate change projections to assess future farming systems and crop production to feed 8.X billion people in 2024 and XX billion people in the 2050, while providing a healthy and liveable environment.