\chapter*{Summary}
\addcontentsline{toc}{chapter}{Summary}
\markboth{SUMMARY}{ }
%\renewcommand{\chaptermark}[1]{ \markright{ \MakeUppercase{#1} } }

Winter wheat (\textsl{Triticum aestivum}) is one of the most important staple crops, cultivated on 20\% of the world's arable land. Ensuring that wheat production keeps up with the growing demand for food and biomass, while reducing its environmental impact and increasing its resilience to a changing environment, is one of the most critical challenges facing agriculture today.

For these reasons, it is important to study plant growth and development at larger spatial scales to improve our understanding of plant-environment interactions and the factors influencing crop productivity. To date, this has been the main motivation for two scientific disciplines: Field phenotyping and remote sensing of vegetation. Field phenotyping is mainly limited to small spatial scales and experimental designs to quantify plant phenotype as a result of genotype-environment interactions. In contrast, remote sensing of vegetation works at large spatial scales to global to describe phenotypes, but often lacks detailed genotype knowledge and experimental design to disentangle different environmental factors. There is therefore a strong case for combining both disciplines to provide a quantitative, physiologically accurate description of plant growth and development for whole agricultural landscapes.

To achieve this, this thesis has developed an open-source prototype system for quantifying winter wheat growth and development at a landscape-scale. The phenotyping prototype combines field phenotyping with space-based optical remote sensing. Here, the term phenotyping refers to the simultaneous recording of plant traits and environmental covariates. The central hypothesis is that field phenotyping can provide prior knowledge of crop phenology and physiology, which can enhance the accuracy and traceability of satellite-based crop trait retrieval.

To construct the prototype, the open-source software Earth Observation data analysis library (EOdal) was developed. EOdal enables the integration of remote sensing imagery, environmental covariates, and field phenotyping data in a reproducible manner. Additionally, field phenotyping data from various sources were compiled for calibration purposes. Furthermore, independent data on growth and phenology-related traits were collected for validation in the Swiss Plateau in 2022 and 2023. The prototype itself consists of three components:

Firstly, an existing process-based phenological model was used to quantify the development of winter wheat, specifically the timing of heading, which is crucial for assessing exposure to post-anthesis stressors like heat. The model was calibrated using 18 years of field phenotyping data and applied to the entire Swiss Plateau over a span of five decades using high-resolution meteorological data. In addition to confirming the model's reliability in estimating heading dates, the findings emphasized the necessity of accurate phenology estimation on larger spatial scales due to topographical variations, annual weather patterns, and the significant impact of climate change, which shifted heading dates by up to 14 days between 1972 and 2020.

Secondly, Sentinel-2 optical satellite imagery was utilized to quantify crop growth conditions at a high spatial resolution. Field phenotyping data were utilized to constrain the retrieval of crop traits from Sentinel-2 imagery through radiative transfer model inversion. The study demonstrated that field phenotyping data can offer prior knowledge of phenology and physiology, resulting in improved retrieval of Green Leaf Area Index (GLAI) and leaf and canopy chlorophyll content from Sentinel-2 imagery. However, these trait estimates must be accompanied by a statement of their uncertainty. As a result, a Monte Carlo framework was developed to quantify uncertainties arising from the radiometry of the Sentinel-2 Multispectral Imager instrument. This ensures the traceability of the results and enables the assessment of their suitability for specific purposes, such as reconstructing crop growth. The relative standard uncertainty in GLAI, for example, could be as high as 5\%. Overall, the results were consistent in terms of spatial patterns, including field heterogeneity, various geographical locations, and different years. Nonetheless, existing radiative transfer models should be refined concerning the representation of vertical gradients and leaf levels.

In the third component, crop growth was reconstructed during the stem elongation period for winter wheat using individual Sentinel-2 observations. Crop growth was formulated as a function of air temperature utilizing dose-response curves, which capture the physiological relationship between air temperature and crop growth. The adoption of dose-response curves facilitated a more physiologically meaningful reconstruction of GLAI time series compared to logistic models widely used in remote sensing. However, dose-response curves require a large calibration dataset with multiple site-year combinations to encompass extreme conditions and the full variability of crop growth conditions. Further efforts should therefore be made to simultaneously assess crop growth, phenology, and environmental covariates of winter wheat varieties. 

With this prototype, it will be possible to quantify crop growth and development at a landscape scale to identify the key drivers of crop productivity. Addressing the complex challenge of sustaining crop production in a changing environment requires the sharing and integration of data, methods and knowledge across disciplines. This thesis demonstrated that the combination of field phenotyping and remote sensing can contribute to agricultural research and applications. Only through this integration will we be able to fully understand the factors influencing plant growth and contribute to agricultural decision making, policy advice and global food security. This thesis provided the scientific and methodological background and proof of concept for the added value of interdisciplinary research in phenotyping and remote sensing. Interdisciplinary research will therefore play an important role in providing sufficient and healthy food today, tomorrow and for all.