\chapter*{Summary}
\addcontentsline{toc}{chapter}{Summary}
\markboth{SUMMARY}{ }
%\renewcommand{\chaptermark}[1]{ \markright{ \MakeUppercase{#1} } }

Today's agriculture faces many challenges. Agricultural production must meet ever-increasing demand for food and biomass while becoming more resource efficient, resilient to climate change and extreme weather events, and reducing harmful environmental impacts. In addition, global food security depends on the ability of a few crops to produce stable yields, among which winter wheat (\textsl{Triticum aestivum}) plays an important role, being grown on 20\% of the world's arable land and accounting for 29\% of the world's cereal yield. Ensuring these crops keep pace with growing demand while adapting crop production to a changing environment is therefore critical.

For these reasons, it is important to study crop growth and development at larger spatial scales to improve our understanding of plant-environment interactions. To date, this has been the main driving force behind two scientific disciplines: Field phenotyping and remote sensing of vegetation. These two disciplines share similar goals and methods, but differ in their focus, objectives, definitions and spatial scales. Field phenotyping is mainly limited to small spatial scales and experimental designs to quantify the crop phenotype as a result of genotype-environment interactions. Remote sensing of vegetation, on the other hand, operates at large spatial scales up to global level to provide descriptions of phenotypes, but often lacks the detailed knowledge of genotypes and experimental design to disentangle different environmental factors. There is therefore a strong case for using concepts and methods from both disciplines to provide a quantitative, physiologically precise description of crop growth and development for whole agricultural landscapes.
The thesis therefore developed an open source prototype system for quantifying winter wheat growth and development at a landscape scale by combining field phenotyping and space-based optical remote sensing. The underlying research questions of accuracy, traceability, spatial up-scaling and physiological plausibility have been addressed in five separate manuscripts that form the chapters of the thesis. Three of these have already been published in peer-reviewed journals. The prototype development focused on the stem elongation period of winter wheat, as this developmental phase is critical for grain yield formation. The geographical focus of the thesis was the Swiss Plateau (Mittelland), which is a blueprint for intensively farmed areas throughout the northern hemisphere.

The thesis starts with a general introduction in Chapter \ref{chap:introduction}, which not only summarises the main motivation, theoretical background and state-of-the-art applications, but also lists the main research questions and objectives.
An open source Python software called Earth Observation data analysis library (EOdal) has been developed to allow unified, reproducible analysis of data streams from satellite remote sensing, field phenotyping and environmental data. This software, used to build the landscape-scale prototype, is described in more detail in Chapter \ref{chap:eodal}.
In Chapter \ref{chap:phemology}, a process-based phenological model was used to quantify the development of winter wheat - more specifically, the timing of heading, which is important for assessing exposure to post-anthesis stressors such as heat. The model was calibrated using field phenotyping data and applied to the entire Swiss Plateau for nearly five decades using high-resolution meteorological data. In addition to confirming that the model is a reliable estimator of heading dates, the results showed the need for accurate phenology estimation at larger spatial scales due to the effects of topography and differences in annual weather patterns, as well as the significant impact of climate change, which shifted heading dates by up to 14 days between 1970 and 2020.
Chapters \ref{chap:uncertainty} to \ref{chap:drc} then focus on the use of field phenotyping data as prior knowledge to improve the retrieval of physiological crop traits such as the Green Leaf Area Index (GLAI) from Sentinel-2 optical satellite imagery. To meet the requirement for traceability, a Monte Carlo method for propagating uncertainty arising from the radiometry of the Sentinel-2 optical imaging instrument is presented in Chapter \ref{chap:uncertainty}. The results show that the uncertainty in the measured radiance values translated into a relative standard uncertainty in GLAI of up to 5\% and up to a few days for satellite-derived phenological metrics. This is important to keep in mind for the subsequent refinement of GLAI and phenology retrieval presented in Chapter \ref{chap:insights}. Here it was clearly shown that field phenotyping data can be used as prior knowledge to improve the retrieval of GLAI and canopy chlorophyll content from Sentinel-2 imagery by imposing physiological and phenological constraints. The traits were retrieved consistently with respect to spatial patterns, e.g. field heterogeneity, different geographical locations and years. Building on these trait estimates, prior knowledge from field phenotyping was then also used to reconstruct crop growth using dose-response curves as described in Chapter \ref{chap:drc}. These curves encoded the relationship between growth, e.g. increase in GLAI over time, and air temperature. This allowed physiologically plausible trait time series to be reconstructed from single satellite observations using a probabilistic data assimilation scheme that combined the high temporal resolution of the air temperature records with the spatial detail of the Sentinel-2 images. In addition, the assimilation scheme took into account the aforementioned uncertainties in the data.
Finally, Chapter \ref{chap:general-discussion} provides a general discussion, including a description of how the individual components of the previous five chapters are combined into the landscape-scale prototype, together with a discussion of its potential, limitations and further research directions.

The main achievement of this thesis is not only a first step towards the unification of field phenotyping and remote sensing, and the spatial up-scaling of growth and development traits from the plot to the landscape scale, but also the design of an open source system of data, software and research results that will allow the simultaneous quantification of crop growth and development. Such a system can make an important contribution to agricultural research, crop breeding and farming practice to provide sufficient and healthy food today, tomorrow and for all.