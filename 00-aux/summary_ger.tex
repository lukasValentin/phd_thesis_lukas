\chapter*{Zusammenfassung}
\addcontentsline{toc}{chapter}{Zusammenfassung}
\markboth{ZUSAMMENFASSUNG}{ }
%\renewcommand{\chaptermark}[1]{ \markright{ \MakeUppercase{#1} } }

Winterweizen (\textsl{Triticum aestivum}) ist eine der wichtigsten Kulturpflanzen und wird weltweit auf 20\% der Ackerfläche angebaut. Eine der grössten Herausforderungen der heutigen Landwirtschaft besteht darin, die Weizenproduktion mit der steigenden Nachfrage nach Nahrungsmitteln und Biomasse in Einklang zu bringen, die Umweltauswirkungen gleichzeitig zu verringern und die Anpassungsfähigkeit an eine sich verändernde Umwelt zu erhöhen.

Aus diesen Gründen ist es wichtig, das Wachstum und die Entwicklung von Pflanzen auf grösseren räumlichen Skalen zu untersuchen, um unser Verständnis der Wechselwirkungen zwischen Pflanze und Umwelt sowie der Faktoren, die die Ertragsleistung von Nutzpflanzen beeinflussen, zu verbessern. Bisher war dies die Hauptmotivation für zwei wissenschaftliche Disziplinen: Feldphänotypisierung und Fernerkundung der Vegetation. Die Feldphänotypisierung beschränkt sich hauptsächlich auf kleine räumliche Skalen und experimentelle Designs zur Quantifizierung des Pflanzenphänotyps als Ergebnis von Genotyp-Umwelt-Interaktionen. Im Gegensatz dazu arbeitet die Vegetationsfernerkundung auf grossen räumlichen Skalen bis hin zur globalen Ebene, um Phänotypen zu beschreiben, verfügt aber oft nicht über detaillierte Kenntnisse der Genotypen und experimentelle Designs, um verschiedene Umweltfaktoren zu entflechten. Es bestehen daher gute Gründe, beide Disziplinen zu kombinieren, um eine quantitative, physiologisch genaue Beschreibung von Pflanzenwachstum und -entwicklung für ganze Agrarlandschaften zu liefern.

Um dieses Ziel zu erreichen, wurde in dieser Arbeit ein Open-Source-Prototyp zur Quantifizierung des Wachstums und der Entwicklung von Winterweizen auf Landschaftsebene entwickelt. Der Phänotypisierungsprototyp kombiniert Feldphänotypisierung mit satellitengestützter optischer Fernerkundung. Unter Phänotypisierung wird die gleichzeitige Erfassung von Pflanzenmerkmalen und Umweltkovariablen verstanden. Die zentrale Hypothese ist, dass die Feldphänotypisierung Vorwissen über die Phänologie und Physiologie von Nutzpflanzen liefern kann, was die Genauigkeit und Rückverfolgbarkeit der satellitengestützten Erfassung von Pflanzenmerkmalen auf grossen Skalen verbessern kann.

Zur Erstellung des Prototyps wurde die Open Source Software Earth Observation data analysis library (EOdal) entwickelt. EOdal ermöglicht die Integration von Fernerkundungsdaten, Umweltkovariablen und Feldphänotypisierungsdaten auf reproduzierbare Weise. Zusätzlich wurden Feldphänotypisierungsdaten aus verschiedenen Quellen zu Kalibrierungszwecken gesammelt. Darüber hinaus wurden unabhängige Daten zu Wachstum und phänologischen Eigenschaften in den Jahren 2022 und 2023 zur Validierung gesammelt. Der Prototyp besteht aus drei Komponenten:

Zunächst wurde ein bestehendes prozessbasiertes phänologisches Modell verwendet, um die Entwicklung von Winterweizen zu quantifizieren, insbesondere den Zeitpunkt des Ährenschiebens, der für die Beurteilung der Exposition gegenüber Nachblüte-Stressoren wie Hitze entscheidend ist. Das Modell wurde mit Feldphänotypisierungsdaten aus 18 Jahren kalibriert und mit hochauflösenden meteorologischen Daten aus fünf Jahrzehnten auf das gesamte Schweizer Mittelland angewendet. Neben der Bestätigung der Zuverlässigkeit des Modells für die Schätzung des Zeitpunktes des Ährenschiebens unterstreichen die Ergebnisse die Notwendigkeit einer genauen Schätzung der Phänologie auf grösseren räumlichen Skalen aufgrund von topographischen Unterschieden, jährlichen Wettermustern und dem signifikanten Einfluss des Klimawandels, der die Ährenschieben zwischen 1972 und 2020 um bis zu 14 Tage verschob.

In der zweiten Komponente wurden optische Sentinel-2 Satellitendaten verwendet, um die Wachstumsbedingungen von Nutzpflanzen in hoher räumlicher Auflösung (10 m) zu quantifizieren. Feldphänotypisierungsdaten wurden verwendet, um die Ableitung von Pflanzeneigenschaften aus Sentinel-2 Bildern durch die Inversion eines Strahlungstransfermodells zu begrenzen. Die Studie zeigte, dass Feldphänotypisierungsdaten Vorwissen über Phänologie und Physiologie liefern können, das zu einer verbesserten Erfassung des Green Leaf Area Index (GLAI) sowie des Blatt- und Bestandeschlorophyllgehalts aus Sentinel-2-Bildern führt. Die abgeleiteten Pflanzenmerkmale müssen jedoch mit einer Aussage über ihre Unsicherheit versehen werden. Zu diesem Zweck wurde ein Monte-Carlo-Ansatz entwickelt, um die Unsicherheiten zu quantifizieren, die durch die Radiometrie des Sentinel-2 Multispektral Imager Instruments entstehen. Dies gewährleistet die Nachvollziehbarkeit der Ergebnisse und ermöglicht die Bewertung ihrer Eignung für spezifische Zwecke wie die Rekonstruktion des Pflanzenwachstums. Zum Beispiel kann die relative Standardunsicherheit des GLAI bis zu 5\% betragen. Insgesamt waren die Ergebnisse in Bezug auf räumliche Muster, einschliesslich Feldheterogenität, verschiedene geographische Standorte und Jahre, konsistent. Dennoch sollten bestehende Stahlungstransfermodelle im Hinblick auf die Darstellung der vertikalen Gradienten und Blattetagen verbessert werden.

Im dritten Teil wurde das Pflanzenwachstum von Winterweizen während des Schossens mit Hilfe von aus Sentinel-2 Beobachtungen abgeleitetem GLAI in stündlicher und täglicher Auflösung rekonstruiert. Das Pflanzenwachstum wurde als Funktion der Lufttemperatur formuliert, wobei Dosis-Wirkungs-Kurven verwendet wurden, um die physiologische Beziehung zwischen Lufttemperatur und Pflanzenwachstum zu erfassen. Die Verwendung von Dosis-Wirkungs-Kurven ermöglichte eine physiologisch sinnvollere Rekonstruktion der GLAI-Zeitreihen im Vergleich zu den in der Fernerkundung weit verbreiteten logistischen Modellen. Allerdings erfordern Dosis-Wirkungs-Kurven eine grosse Menge an Kalibrierungsdaten mit mehreren Standort-Jahres-Kombinationen, um extreme Bedingungen und die volle Variabilität der Wachstumsbedingungen abzudecken. Daher sollten weitere Anstrengungen unternommen werden, um das Pflanzenwachstum, die Phänologie und die Umweltfaktoren von Winterweizensorten gleichzeitig zu bewerten.

Der in dieser Studie entwickelte Prototyp ermöglicht die Quantifizierung des Wachstums und der Entwicklung von Nutzpflanzen auf Landschaftsebene und erleichtert somit die Identifizierung zentraler Faktoren der Ernteertragsbildung. Die Bewältigung der komplexen Herausforderung des nachhaltigen Pflanzensbaus in einer sich verändernden Umwelt erfordert den Austausch und die Integration von Daten, Methoden und Wissen aus verschiedenen Disziplinen. Diese Arbeit hat das Potenzial der Kombination von Feldphänotypisierung und Fernerkundung für die landwirtschaftliche Forschung und Anwendungen aufgezeigt. Durch die Integration dieser Ansätze können wir die Faktoren, die das Pflanzenwachstum beeinflussen, besser verstehen und fundierte Entscheidungen über landwirtschaftliche Praktiken, Politik und globale Ernährungssicherheit treffen. Diese Arbeit liefert die wissenschaftliche und methodologische Grundlage sowie den Machbarkeitsnachweis für den Wert interdisziplinärer Forschung in der Phänotypisierung und Fernerkundung. Daher wird interdisziplinäre Forschung eine entscheidende Rolle spielen, um eine ausreichende und gesunde Ernährung für die gegenwärtigen und zukünftigen Generationen sicherzustellen.