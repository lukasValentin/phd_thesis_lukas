\chapter*{Acknowledgements}
\addcontentsline{toc}{chapter}{Acknowledgements}
\markboth{ACKNOWLEDGEMENTS}{ }
\renewcommand{\sectionmark}[1]{ \markright{ \MakeUppercase{#1} } }

It takes about eight months from sowing to harvesting winter wheat. Writing a PhD thesis on the growth and development of winter wheat takes a little longer, or -- in the case of winter wheat -- up to three growing seasons. Of course, like growing winter wheat, writing a PhD thesis requires a lot of external input and support from many people, whom I would like to thank here for their commitment.

First of all, I would like to thank my two supervisors, Helge Aasen and Achim Walter, for giving me the opportunity and freedom to work on the PhD thesis within the PhenomEn project, but also for their valuable input, discussions and feedback on my work. However, a PhD thesis requires not only supportive supervisors, but also the inspiring exchange with fellow PhD students, namely Quirina Merz, Corina Oppliger, Olivia Zumsteg, Mike Boss, Nicolin Caflisch, Matthias Diener, Joaquin Gajardo Castillo, Gregor Perich, Simon Treier and Flavian Tschurr, and senior researchers who also critically challenged my ideas, namely Jonas Anderegg, Andreas Hund, Beat Keller, Afef Marzougui, Norbert Kirchgessner, Lukas Roth, and Nicola Storni. I would also like to thank the entire Crop Science group at ETH, especially Marianne Wettstein, for her kind words and help with all administrative matters, and the Earth Observation of Agroecosystems team at Agroscope Reckenholz, especially Miguel Kohling and Fabio Orani, as well as Francesco Argento and Raphael Portmann, for their software beta testing, provision of data, and critical listening and review.

Not forgetting all the helping hands in the field, in the laboratory and the support of all those who gave me access to their field sites and allowed us to collect all the data needed to build and validate the models. My special thanks go to Florian Abt and Thomas Anken and his team at Agroscope Tänikon and the Swiss Future Farm, Marco Landis at Strickhof, Bernhard Streit and Fred Burri for providing access and resources at Witzwil, and to my student assistants, namely Bérénice Goin, Karia Kögler, Daria Larcher, Jania Mackenthun, Vilma Rantanen, Dea Spiess, Rike Teuber, Finn Timcke and Samuel Wildhaber. I would also like to thank the Agroscope field technicians Katharina Casada, Karin Meier-Zimmermann, Matthias Hatt and Hansulrich Zbinden for their support in the field and with the freeze dryer.

Outside of Agroscope and ETH, I would like to thank my many contacts and collaborators from the SENSECO COST Action, with whom I was privileged to spend some exciting workshops. In particular, I would like to thank Eatidal Amin, Clement Atzberger, Katja Berger, Dessislava Ganeva, Tobias Hank, Andreas Hueni, Lammert Kooistra, Egor Prikaziuk, Hanna Sjulgård, Stefanie Steinhauser, and Jochem Verrelst for the enriching exchange and collaboration. A big thank you also goes to Javier Gorro\~{n}o, without whom I would hardly have mastered the uncertainty calculation.

As a PhD thesis is not only about science but also about the financial resources to do exciting research, I have to thank the Swiss National Science Foundation for funding the PhenomEn project and the COST Association for supporting my travels to SENSECO summer schools, meetings and workshops.

Last but not least, I would like to thank my former supervisors and teachers, Stefan Lang and Dirk Tiede, without whom I would not have even noticed the call for applications for the PhD position at the ETH.

Finally, a huge thank you to my flatmates, friends and family who have always had an open ear, warm words and advice for me.

\bigskip
\centering
\textit{Thank you so much! Danke vielmals! Merci beaucoup! Grazie mille! Grazia!}
