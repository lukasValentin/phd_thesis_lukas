\chapter*{Acknowledgements}
\addcontentsline{toc}{chapter}{Acknowledgements}
\markboth{ACKNOWLEDGEMENTS}{ }
\renewcommand{\sectionmark}[1]{ \markright{ \MakeUppercase{#1} } }

It takes about eight months from sowing to harvesting winter wheat. Writing a PhD thesis on the growth and development of winter wheat takes a little longer, or - in the case of winter wheat - up to three growing seasons. Of course, like growing winter wheat, writing a PhD thesis requires a lot of external input and support from many people, whom I would like to thank here for their commitment.

First of all, I would like to thank my two supervisors, Helge Aasen and Achim Walter, for giving me the opportunity and freedom to work on the PhD thesis within the PhenomEn project, but also for their valuable input, discussions and feedback on my work. However, a PhD thesis requires not only a supportive supervisor, but also the inspiring exchange with fellow PhD students, namely Quirina Merz, Corina Oppliger, Olivia Zumsteg, Mike Boss, Nicolin Caflisch, Matthias Diener, Joaquin Gajardo Castillo, Gregor Perich, Simon Treier, and Flavian Tschurr, and senior researchers who also critically challenged my ideas, namely Jonas Anderegg, Andreas Hund, Beat Keller, Afef Marzougui, Norbert Kirchgressner, Lukas Roth and Nicola Storni. I would also like to thank the entire Crop Science group at the ETH, especially Marianne Wettstein, for her kind words and help with all administrative matters, and the Earth Observation of Agroecosystems group at Agroscope Reckenholz, especially Miguel Kohling and Fabio Orani, for their software beta testing and critical listening and reviewing.

% thanks to my supervisors, the group of crop science (Norbert, Beat, Andi), EOA-team Reckenholz (Miguel, Fabio)
% thanks to Hiwis and technicians
% thanks to SNF and COST
% thanks to Andi Hueni
