\subsection{Time series reconstruction accuracy and plausibility}
Although the raw \gls{GLAI} values and the reconstructed \gls{GLAI} are not directly comparable due to the different number of data points, we conclude that the reconstructed \gls{GLAI} values using \gls{DRC}s and the baseline reduced the \gls{GLAI} retrieval error (Figures \ref{fig:s2-obs-scatter-plots} and \ref{fig:glai-scatter-plots}). This was mainly due to the removal of outliers in the negative y-direction caused by atmospheric perturbations, suggesting that both the \gls{DRC} and baseline approaches dealt reasonably well with the effects of undetected clouds and cloud shadows. Nevertheless, a systematic underestimation of GLAI values greater than 5 $m^2$ $m^{-2}$ was observed for the \gls{GLAI} baseline. This underestimation was hardly noticed in the proposed reconstruction with \gls{DRC}s (see Figure \ref{fig:glai-scatter-plots}) as the \gls{DRC} \gls{GLAI} was mostly higher than the baseline (Figures \ref{fig:glai-trajectories} and \ref{fig:model-intercomparison}). The underestimation of \gls{S2} \gls{GLAI} observations was probably due to the \gls{RTM} inversion approach used: It is a known problem that \gls{RTM}s such as PROSAIL exhibit saturation phenomena at high biomass levels due to leaf clumping~\citep{richter_evaluation_2011}. As the baseline only uses the raw \gls{S2} \gls{GLAI} observations, the fit could not compensate for saturation effects, so the reconstructed time series consequently underestimated \gls{GLAI}. In addition, the sigmoid fit aims to minimise the mean error of the reconstructed curve to the raw \gls{S2} \gls{GLAI} observations. This may lead to further underestimation of \gls{GLAI} values, as the reconstructed curve may sometimes be lower than the underlying \gls{S2} \gls{GLAI} observations.

In the case of \gls{DRC}s, the assimilation scheme integrates two data sources with distinct advantages: The \gls{DRC}s contain prior physiological knowledge about the relationship between air temperature and growth, thereby mitigating the underestimation of \gls{GLAI} values as this relationship was established using high-quality in-situ data. The raw \gls{S2} \gls{GLAI} provides spatial details that are absent from the temperature data. This makes the approach well-suited for fine-grained spatial growth analysis. In addition, as air temperature records are usually continuous, the \gls{GLAI} reconstruction between \gls{S2} observations relies on encoded physiological knowledge, reducing the likelihood of unrealistically fast growth rates due to physiological constraints imposed by the temperature. It is not ensured that the baseline will accurately reflect the prevailing conditions. This is due to the fact that reconstruction between \gls{S2} observations solely relies on the function parameters, which do not necessarily contain sufficient information about the underlying biological mechanisms. Consequently, the baseline might indicate high growth rates even if the temperature is significantly below or above the critical $T_{min}$ and $T_{max}$ thresholds.

The accuracy of the \gls{DRC}-reconstructed \gls{GLAI} was comparable to approaches using more complex mechanistic crop growth models, which require a significantly higher number of parameters: \cite{ma_wheat_2022} reported values of $R^2$ between 0.7 and 0.73 for winter wheat in northern China (relative errors between 22 and 26\%) using the \gls{SAFYE} crop growth model in combination with \gls{S2} images for two growing seasons. This is comparable to the accuracy using \gls{DRC}s (Table \ref{tab:error-stats}). Higher accuracy was reported by \cite{hank_using_2015} for winter wheat in southern Germany. They achieved a root mean square error of 0.35 $m^2$ $m^{-2}$ ($R^2$ 0.96) using a more complex crop growth model combined with Landsat and RapidEye satellite remote sensing data. However, their sample size was small (N = 19) and included only a single growing season and field parcel. Even smaller errors were reported by \cite{zhang_improving_2021} (relative errors between 2.0 and 9.2\%) using \gls{SAFYE} for two growing seasons of winter wheat in central China. Instead of using satellite imagery, they used \gls{GLAI} retrieved from handheld hyperspectral data, which is arguably not comparable to space-borne \gls{GLAI} retrieval. However, more complex crop growth models often aim to model phenology or even yield, whereas the approach presented is designed to interpolate \gls{GLAI} observations in a physiologically meaningful way. This also means that the reduced complexity, and perhaps accuracy, can be compensated for by using the \gls{GLAI} observations as guidance over the growing season.

However, the \gls{DRC} approach is also likely to be limited by the lack of spatial detail during long periods without \gls{S2} passes due to cloud cover -- a problem shared with more complex crop growth models. Assimilation includes information on crop growth that has causes other than temperature alone, such as differences in soil properties or subtle differences in management. Without regular assimilation, this information cannot be incorporated into the \gls{DRC} growth rates, limiting the accuracy of comparing the reconstructed \gls{GLAI} with in-situ data. Therefore, a higher number of \gls{S2} observations is likely to result in higher reconstruction accuracy. This means that increasing the number of observations, e.g. by fusing \gls{GLAI} from cube satellite constellations as suggested by \cite{sadeh_fusion_2021}, could further increase the reconstruction accuracy. This method has two major drawbacks: First, the amount of data and model complexity increases significantly due to the addition of a second satellite platform. One of the main advantages of the \gls{DRC} approach, however, is its simplicity. Secondly, most cube satellite constellations, unlike \gls{S2}, are commercial products that carry a financial burden that not all users of remote sensing data may be able to bear. Still, as the question of the optimal number of satellite observations and their temporal distribution for data assimilation does not seem to have been conclusively clarified, there is potential for further research.

Of the three \gls{DRC}s utilised, Wang Engels exhibited minimal bias, albeit the most inconsistent year-on-year outcome (see Tables \ref{tab:error-stats}-\ref{tab:error-stats-years}). This is significant as the Wang Engels \gls{DRC} has the most physiological significance, thereby making it a suitable candidate to examine the impact of rising temperatures and stress factors in the study area \citep{tschurr_climate_2020}. Since there is a lack of additional in-situ \gls{GLAI} data, the optimal approach was to optimize the Wang Engels \gls{DRC} using only three parameters. However, with additional data at hand, the year-to-year error could potentially decrease by optimizing an extra parameter without overfitting the data. In order to achieve this, a scaling parameter could be integrated, offering another degree of freedom to optimize $T_{base}$, $T_{opt}$, and $T_{max}$. Consequently, the Wang Engels \gls{DRC} \gls{GLAI}'s performance could possibly be enhanced with more calibration data accessible. For now, the asymptotic \gls{DRC} seems to be the most suitable choice: It is more sophisticated and marginally more precise than the nonlinear \gls{DRC}. Moreover, its year-to-year performance is steady. Again, it is worth mentioning that additional in-situ calibration data from other environments (site-year combinations) would be advantageous for making a conclusive statement about selecting the \gls{DRC} and studying the year-to-year performance and performance within selected phenological stages (Figure \ref{fig:glai-errors-phenology}).

Concerning the selection of the temporal resolution of the air temperature data, our results did not reveal any pronounced tendency (see Table \ref{tab:error-stats}). Finer resolved covariate measurements could theoretically offer more information and therefore enhance growth prediction accuracy from a physiological standpoint. However, daily air temperature data is more accessible and requires fewer computational resources from an operational perspective. Overall, a conclusive answer to the second research question cannot be provided. Considerations related to physiology suggest that the use of hourly air temperature data is more favorable than daily data. As argued before, further calibration and validation data would be necessary to arrive at a conclusive statement.

\subsection{Time series reconstruction stability}
The baseline \gls{GLAI} resulted in up to 20\% of pixels for which no \gls{GLAI} time series could be reconstructed (Figure \ref{fig:maps-baseline-failure}). This is due to the lack of a sufficient number of raw \gls{S2} \gls{GLAI} observations or non-convergence of the optimiser (Levenberg-Marquardt, section \ref{subsec:glai-reconstruction-sucess}). Increasing the number of iterations could counteract the non-convergence problem. The choice of the initial guess is also important for the successful and fast convergence of the optimiser. Still, there is no guarantee that the optimiser will converge and find a global minimum.

It could be argued that the absence of up to 20\% of pixels might not significantly impact the results of aggregate statistics (such as median \gls{GLAI} values per field parcel) in large-scale analyses where sub-field heterogeneity is negligible. However, we maintain that two issues persist.

First and foremost, spatial gaps in the reconstructed \gls{GLAI} may result in inadequate sub-field scale analyses, particularly for precision farming applications. The same applies to small-scale farming systems with small field sizes (< 1 ha), for which the share of missing pixels might easily reach up to 100\% due to the small number of \gls{S2} pixels covering a parcel.

Secondly, there are significant gaps within the field that are frequently the result of single observations being masked out by scene pre-classification. As previously discussed, the \gls{S2} \gls{SCL} typically proves unreliable in accurately delineating clouds and shadows. Therefore, atmospheric disturbances may well have affected the neighbouring pixels, for which GLAI reconstruction proved successful from a technical point of view. Still, the pixels may exhibit physiologically implausible growth patterns as a result of the partially degraded quality of the original \gls{S2} \gls{GLAI} observations. The degenerated quality of the input data cannot be sufficiently compensated without the corrective effect of the \gls{DRC}-based growth curves. We maintain that our suggested method surpasses statistical time series reconstruction in terms of reliability, as stated in our second research question.

\subsection{Implications for crop productivity assessment}

The underestimation of \gls{GLAI} values by the baseline has significant consequences for the assessment of crop productivity based on remote sensing, which often relies on methods similar to the baseline~\citep{kooistra_reviews_2023}. This issue is exemplified by \gls{GPP}, an indicator of energy fixed by photosynthesis minus losses through photorespiration \citep{hilty_plant_2021}, which is also used on a global scale to study the effects of climate change on plant growth~\citep{campbell_large_2017}. To estimate crop canopy \gls{GPP} from remote sensing data, \gls{LUE} models are often used \citep[for instance]{dong_deriving_2017}. These models describe the efficiency with which \gls{PAR} is converted into photosynthesis. As \cite{monsi_factor_2004} demonstrated, the fraction of \gls{PAR} intercepted by a canopy is linearly correlated with \gls{GLAI}. Thus, according to \citet{gitelson_productivity_2015}, precise estimates of \gls{LUE} and \gls{GLAI} are crucial for accurate estimation of \gls{GPP} at canopy level. If maximum GLAI values are systematically underestimated, as in the case of raw and baseline GLAI, this could potentially affect the determination of GPP. To improve the accuracy and reliability of remotely sensed GPP estimates, our proposed method may be suitable. However, it is important to remember that \gls{GPP} estimates do not only depend on \gls{GLAI} and that the linear relationship between light interception and \gls{GLAI} only holds true under the assumption of an idealized turbid medium which might fail for heterogeneous canopies \citep{hilty_plant_2021}. Therefore, a more detailed assessment would be required to provide a quantitative estimate of the impact of underestimated \gls{GLAI} on estimates of \gls{GPP} or biomass. However, this is beyond the scope of this paper and should be addressed in further research.

In addition, the probabilistic data assimilation scheme accounts for model and data uncertainties, resulting in improved accuracy. The quantification of uncertainty is critical because it allows users to determine the suitability of a data product, such as the reconstructed \gls{GLAI} time series, for a particular purpose, such as yield estimation as a measure of crop productivity. This information is not available from the baseline. In addition, the reported uncertainty can be transferred to derived products, adding further value. This is important in the context of decision support for adaptive crop management and could lead to more informed agricultural decision making~\citep{meenken_bayesian_2021}.

\subsection{Ways forward}

The utilisation of prior knowledge about physiological processes holds the potential to enhance contemporary agricultural remote sensing methods. To bolster the reliability of our presented model, expansion of the calibration dataset to encompass more environments would be advantageous. This up-scaling would augment our ability to establish the temperature bounds ($T_{min}$ and $T_{max}$) which regulate crop growth. This is especially important in the case of more advanced \gls{DRC}s like Wang Engels, which revealed promising performance due to its low bias (Table \ref{tab:error-stats}). Furthermore, the dataset at hand demonstrated an imbalance in the measurements per site, which could potentially impact the final results. The absence of publicly accessible in-situ records evaluating phenology, \gls{GLAI} measurements, and temperature is preventing the expansion of the dataset at present. Nevertheless, the ground truth data proved adequately representative to parameterise the \gls{DRC} curves shown and to outperform the baseline method. As a result, we propose that upcoming field trials should include phenology and a minimum of environmental variables, along with functional crop characteristics, to facilitate development of physiological models. This will enable more rigorous parameter optimization and lead to a reduction in \gls{RMSE}. Furthermore, it may be possible to estimate traits like yield while avoiding the use of complex crop growth models.

Regarding phenology, the approach could be expanded to encompass the entire phenological development cycle of wheat. In order to achieve this, sufficient calibration data is required for the phenological macro-stages preceding and following the stem elongation period, including the tillering or senescence phase. A phenology model is thus necessary for determining the timing and duration of phenological development stages. Such a phenology model should ideally describe the entire phenology using a simple and easily applicable approach, such as the \gls{DRC}, which can even combine multiple environmental parameters.

Additionally, meteorological drivers of crop growth, such as vapor-pressure-deficit (VPD) or global radiation, could be included, apart from temperature. These meteorological parameters, however, present greater difficulty in terms of measurement and acquisition. Our proposal utilises air temperature as a readily available meteorological metric, which not only simplifies the approach but also renders it potentially implementable on a global scale. Furthermore, this modelling approach using \gls{DRC} curves can also be applied to other crops \citep{parent_temperature_2012, roth_field_2023}.