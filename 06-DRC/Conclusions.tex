We have demonstrated that the methodology based on \gls{DRC}s, incorporating physiological a-priori knowledge pertaining to crop growth, offers substantial benefits compared to statistical models often used in remote sensing, while avoiding the complexity of mechanistic crop growth models. By integrating temperature, an important environmental driver of plant growth, with raw \gls{S2} \gls{GLAI} observations by an probabilistic data assimilation scheme, we were able to reduce the systematic underestimation of high in-situ \gls{GLAI} values and produce more reliable estimates of crop growth. This approach allowed to preserve the spatial detail of the \gls{S2} data, regard physiological constraints on growth predictions and and quantify uncertainties.

We deduce that integrating a-priori physiological understanding by using dose-response curves boasts tremendous potential for promoting agricultural remote sensing generally and crop productivity estimation, specifically. Based on the growing availability of crop phenotyping datasets, this study can serve to enhance both crop growth modelling and agricultural yield estimation.
