The majority of daily calorie intake is provided by a few arable crops, including wheat. Ongoing climate change poses a major challenge to the ability of such crops to produce resilient yields \citep{asseng_rising_2015}. This calls for suited management practices to mitigate risks and increase the resilience of agroecosystem. Consequently, a sound understanding of plant growth is urgently needed to identify and minimise crop risks \citep{tilman_global_2011}. Plant growth dynamics within different phenological phases can be of great interest to identify stressors \citep{reynolds_physiological_2016}. An important phase with respect to the yield potential of winter wheat (\textsl{Triticum aestivum}) is the stem elongation phase (i.e., begin of stem elongation until begin of flowering), which will be the focus of this study \citep{kronenberg_monitoring_2017, miralles_duration_2000}.

Using optical satellite remote sensing, plant growth can be recorded on large spatial scales with relatively high temporal resolution. Remotely sensed time series of functional crop traits such as \gls{GLAI} -- defined as the photosynthetically active leaf area per unit ground area ~\citep{maddonni_leaf_1996} -- are therefore widely used to estimate vegetation productivity ~\citep{kooistra_reviews_2023}. For time series reconstruction, mainly statistical models are used, which fit a function to a set of satellite observations. Over the past decades, a variety of these statistical reconstruction models have been proposed  \citep{zeng_review_2020, kooistra_reviews_2023}. These models range from simple linear interpolation to models that already incorporate prior knowledge about vegetation development, such as \gls{DL}~\citep{beck_improved_2006}. \gls{DL} take advantage of the fact that most crop traits follow a bell curve with an ascending branch for the generative phase and a descending branch for the senescent phase. \gls{DL} are therefore a clear advancement compared to time series reconstruction methods such as the Savitzky-Golay filter \citep{savitzky_smoothing_1964}, the Whittaker smoother \citep{eilers_perfect_2003}, or Gaussian processes regression \citep{belda_optimizing_2020, pipia_green_2021} that lack a explicit formulation of basic principles of crop growth and development. \gls{DL} can be used to plausibilize the estimation of functional crop traits, i.e., to check whether temporal trajectories are consistent with prior knowledge~\citep{koetz_use_2005}. Strictly speaking, this reconstruction is a modeling of crop growth.

Still, even such advanced models depend on the availability of a sufficiently high number of satellite observations. The number of observations in optical remote sensing, however, can be reduced significantly by unfavorable atmospheric conditions such as clouds. In mid-latitude environments, which represent a major part of the world's wheat production area, the percentage of cloudy optical satellite images can be higher than  60\%~\citep{sudmanns_assessing_2020}. This leads to larger temporal gaps in the data which constrain time reconstruction accuracy~\citep{zhou_reconstruction_2015}. Moreover, undetected clouds and shadows, i.e., noise, can deteriorate the quality of time series reconstruction~\citep{zhou_performance_2016}. This is significant as the reconstruction methods approach crop growth modeling mainly from a statistical perspective, i.e., they make strong assumptions about the distribution and power of signal and noise. Moreover, the model parameters of statistical methods such as the aforementioned Whittaker smoother or Savitzky-Golay filter have often no intrinsic biological or physical meaning. Thus, the physiological plausibility of the reconstructed time series is not guaranteed resulting in a potentially misleading representation of crop growth. Nevertheless, the acceptance of these models in the remote sensing community is high~\citep{kooistra_reviews_2023} as the models are usually fast and easy to use.

A more advanced perspective on crop growth and development is provided by mechanistic crop models that address the underlying physiological processes~\citep{delecolle_remote_1992, jamieson_sirius_1998,keating_overview_2003}. Mechanistic, or process based, models are explicit formulations of physical and biological processes, with physical and biological meaning assigned to all parameters of the model \citep{cox_towards_2006}. However, these models require extensive calibration efforts and information about boundary conditions such as soil properties which are often not available. To address this issue, the assimilation of remotely sensed functional traits has been proposed~\citep{pellenq_methodology_2004} and shown to improve vegetation productivity estimation~\citep{huang_assimilation_2019, waldner_high_2019}. Still, the complexity of mechanistic models and lack of calibration data limit their use in agricultural remote sensing~\citep{weiss_remote_2020} although more simpler models such as the \gls{SAFYE} have been proposed \citep{ma_wheat_2022}.

From a purely physiological perspective, temperature is one of the most important and yet easy to measure covariates controlling plant growth~\citep{porter_temperatures_1999, asseng_climate_2019}. A simple and widely used example in this regard is the concept of \gls{GDD}~\citep{mcmaster_growing_1997}. \gls{GDD} describe the change of a trait value, i.e., growth, as the accumulation of temperature sums. This, however, partly neglects the effect that any chemical and, hence, biological process takes place within a specific temperature range and that reaction (growth) rates are a function of temperature. In detail, there is a minimum or base temperature $T_{base}$ below which no growth occurs as well as a maximum temperature $T_{max}$ above which growth comes to a halt. Between $T_{base}$ and $T_{max}$ there is an optimal temperature, $T_{opt}$, at which the growth rate reaches its maximum~\citep{porter_temperatures_1999}.

Various \gls{DRC}s have been proposed to model growth as a function of temperature \citep{wang_uncertainty_2017}. The range of functions varies from the above \gls{GDD}s to the use of more complex functions such as asymptotic curves \citep{roth_phenomics_2022}, the curve proposed by \cite{wang_simulation_1998} or the Arrhenius-shaped curve proposed in \cite{parent_temperature_2012}. The parameters of the \gls{DRC}s have -- like mechanistic crop models -- a biological meaning, but require only a few parameters, which arguably makes them easy to use. ~\citet{roth_phenomics_2022} have shown that crop growth rates under field conditions can be accurately reconstructed from \gls{DRC}s. The authors have also shown that \gls{DRC}s based on hourly air temperature data allow interpolation of coarser resolution (every three to four days) trait observations. However, to the best of our knowledge, a \gls{DRC}-based time series reconstruction approach has not been used to interpolate between satellite-derived crop trait observations.

Our primary objective is therefore to use a priori physiological knowledge of the dependence of plant growth on air temperature encoded in \gls{DRC}s to improve the reconstruction of \gls{GLAI} time series from a set of satellite observations. We hypothesise that the use of physiologically informed \gls{DRC}s and high spatial resolution trait observations will provide an accurate, physiologically consistent representation of crop growth. We therefore assume \gls{DRC}s to outperform statistical time series reconstruction methods that lack an explicit linkage to biology.

Based on our objective, we formulate three research questions:

\begin{itemize}
    \item First, can \gls{DRC} crop growth rates be used to reconstruct continuous, physiologically plausible crop trait time series from a set of satellite observations?
    \item Second, does the proposed approach outperform a time series reconstruction based on remote sensing data alone in terms of accuracy and reliability?
    \item Third, what temporal resolution of temperature data is required - hourly or daily?

\end{itemize}

To address these questions, we focus on \gls{GLAI} derived from the \gls{S2} satellite constellation at a study region in Switzerland, which acts as a blue-print for intensively farmed agricultural landscapes in temperate climate zones.

We start with a description of the in-situ \gls{GLAI} data used to calibrate and validate our proposed methodology (Section \ref{sec:drc_study-area-data}). We then describe the fitting of the \gls{DRC}s to encode a-priori physiological knowledge. We continue with the \gls{GLAI} retrieval from \gls{S2} to introduce spatial detail and large area coverage, and the proposed probabilistic reconstruction scheme in Section \ref{sec:drc_methods} alongside a baseline method based on \gls{S2} \gls{GLAI} observations, only.